\documentclass[12pt, a4paper]{article}

\usepackage[utf8]{inputenc}
\usepackage[T1]{fontenc}
\usepackage[french]{babel}
\usepackage[official]{eurosym}
\usepackage[left=2cm, right=2cm, top=2.5cm, bottom=2.5cm]{geometry}
\usepackage{helvet}
\renewcommand{\familydefault}{\sfdefault}
\usepackage{tabularx}
\usepackage[table]{xcolor}
\usepackage{booktabs}
\usepackage{enumitem}
\usepackage{fancyhdr}
\usepackage{parskip}
\usepackage{etoolbox} 
\usepackage{xfp}
\usepackage{siunitx}

\sisetup{
    output-decimal-marker = {,},
    group-separator = {\,},
    group-minimum-digits = 4,
    detect-all
}

\setlength{\parindent}{0pt}

\definecolor{InvoiceBlue}{HTML}{006EB4}
\definecolor{InvoiceBlack}{HTML}{333333}
\definecolor{TableHeaderGray}{HTML}{F5F5F5}

\newcommand{\TotalHT}{0}
\newcommand{\ContenuTableau}{}

\newcommand{\numerofacture}{2025-001}
\newcommand{\DateFacture}{01/02/2025}
\newcommand{\DateEcheance}{01/03/2025}
\newcommand{\MontantPaye}{0}
\newcommand{\IBAN}{FR76 XXXX XXXX XXXX XXXX XXXX XXX}
\newcommand{\BIC}{XXXXFRPPXXX}
\newcommand{\RIB}{XXXXX XXXXX XXXXXXXXXXX XX}
\newcommand{\SirenClient}{XXX XXX XXX}
\newcommand{\MailClient}{client@example.com}
\newcommand{\LocaClient}{123 Rue Exemple, 75001 Paris}
\newcommand{\MailPrestataire}{prestataire@example.com}
\newcommand{\DenominationSocialePrestataire}{\textbf{NOM PRESTATAIRE}}
\newcommand{\DenominationSocialeClient}{\textbf{NOM CLIENT}}
\newcommand{\SirenPrestataire}{XXX XXX XXX}
\newcommand{\LocaPrestataire}{456 Avenue Exemple, 75002 Paris}

\newcommand{\AjouterLigne}[4]{%
    \edef\TotalLigne{\fpeval{round(#3 * #4, 2)}}%
    \xdef\TotalHT{\fpeval{round(\TotalHT + \TotalLigne, 2)}}%
    
    \begingroup
    \edef\x{\endgroup
        \noexpand\appto\noexpand\ContenuTableau{%
            \noexpand\textcolor{InvoiceBlue}{\unexpanded{#1}} & 
            \unexpanded{#2} & 
            \noexpand\num{#3} \noexpand\euro{} & 
            \unexpanded{#4} & 
            \noexpand\num{\TotalLigne} \noexpand\euro{} \noexpand\\
            \noexpand\midrule
        }%
    }\x
}

\newcommand{\DescriptionBlock}[2]{%
    \textbf{#1}\par
    \vspace{0.1cm}
    \begin{itemize}[leftmargin=*, noitemsep, topsep=0pt, partopsep=0pt, label={-}]
        #2
    \end{itemize}
    \vspace{0.1cm}
}

\newcommand{\GenererLesLignes}{
    
    \AjouterLigne{Prestation}{%
        \DescriptionBlock{Description de la prestation principale}{
            \item Point détaillé numéro un
            \item Point détaillé numéro deux
        }
    }{1500.00}{4}

    \AjouterLigne{Service}{%
        \DescriptionBlock{Description du service}{
            \item Détail du service fourni
        }
    }{450.00}{2}

    \AjouterLigne{Forfait}{%
        \DescriptionBlock{Forfait mensuel}{
            \item Maintenance
            \item Support technique
        }
    }{200.00}{1}

    \AjouterLigne{Consultation}{%
        \DescriptionBlock{Heures de consultation}{
            \item Analyse des besoins
            \item Recommandations
        }
    }{100.00}{5}
}

\newcommand{\CalculerMontantDu}{
    \edef\MontantDu{\fpeval{round(\TotalHT - \MontantPaye, 2)}}
}

\pagestyle{fancy}
\fancyhf{}
\fancyfoot[C]{\color{InvoiceBlack}\small\textbf{TVA non applicable - article 293 B du CGI}}
\renewcommand{\headrulewidth}{0pt}
\renewcommand{\footrulewidth}{0pt}

\begin{document}
\color{InvoiceBlack}

\GenererLesLignes
\CalculerMontantDu

\begin{tabularx}{\textwidth}{@{} X X @{}}
    \textbf{\MakeUppercase{Émetteur}} \par
    \DenominationSocialePrestataire \par
    SIREN \SirenPrestataire \par
    \MailPrestataire \par
    \vspace{0.2cm}
    \LocaPrestataire
    & 
    \textbf{\MakeUppercase{Destinataire}} \par
    \DenominationSocialeClient \par
    SIREN \SirenClient \par
    \MailClient \par
    \vspace{0.2cm}
    \LocaClient
\end{tabularx}

\vspace{1.2cm}

\begin{tabularx}{\textwidth}{@{} X >{\raggedleft\arraybackslash}X @{}}
    {\fontsize{32}{38}\bfseries FACTURE \numerofacture}
    &
    \begin{tabular}{@{}l r@{}}
        Date & \DateFacture \\
        Échéance & \DateEcheance \\
        \addlinespace
        \textbf{Montant dû} & \textbf{\num{\MontantDu}}\, \euro{} \\
    \end{tabular}
\end{tabularx}

\vspace{1.cm}

\renewcommand{\arraystretch}{1.4}
\begin{tabularx}{\textwidth}{
    p{3.0cm}                
    X                       
    >{\raggedleft}r
    >{\centering}c
    >{\raggedleft\arraybackslash}r
}
    \toprule
    \rowcolor{TableHeaderGray}
    \textbf{Type} & \textbf{Description} & \textbf{P.U. HT} & \textbf{Qté} & \textbf{Total} \\
    \midrule
    
    \ContenuTableau
    
\end{tabularx}
\renewcommand{\arraystretch}{1.0}

\vfill 

\begin{tabularx}{\textwidth}{@{} X >{\raggedleft\arraybackslash}X @{}}
    \begin{tabular}[b]{@{}l@{}}
        \textbf{RIB:} \par \\
        \texttt{IBAN: \IBAN} \par \\
        \texttt{BIC: \BIC } \par \\
        \texttt{RIB: \RIB} \par \\
    \end{tabular}
    
    & 
  
    \begin{tabular}[b]{@{}l r@{}}
        Total HT & \num{\TotalHT} \euro{} \\
        Total net de TVA & \num{\TotalHT} \euro{} \\     
        Payé à ce jour & \num{\MontantPaye} \euro{} \\
        \addlinespace
        \textbf{Montant dû} & \textbf{\num{\MontantDu}} \euro{} \\
    \end{tabular}

\end{tabularx}

\medskip
\end{document}
